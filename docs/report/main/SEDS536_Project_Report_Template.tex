\documentclass[14pt]{article}

\usepackage{graphics}
\usepackage{epsfig}
\usepackage{times}
\usepackage{amsmath}
\usepackage{subcaption}
\usepackage{multirow}
\usepackage{gensymb}
\usepackage{array}
\usepackage{float}
\usepackage{indentfirst}
\newcolumntype{P}[1]{>{\centering\arraybackslash}p{#1}}
\usepackage{booktabs, makecell}
\usepackage{diagbox}
\usepackage{float}
\floatstyle{plaintop}
\restylefloat{table}
\usepackage{natbib}


\topmargin      0.0in
\headheight     0.0in
\headsep        0.0in
\oddsidemargin  0.0in
\evensidemargin 0.0in
\textheight     9.0in
\textwidth      6.5in

\title{{\bf SEDS536 Image Recognition \\ Project Progress Report \\ Spring 2025} \\
	\it Project Title: Handwritten Exam Paper Reader}


\date{\today}

\begin{document}

\pagestyle{plain}
\pagenumbering{roman}
\maketitle


\begin{figure}[H]
	\centering
	\includegraphics[ scale = 0.35]{iyte_logo} 
\end{figure}


\begin{itemize}
	\item Mustafacan Koç - 323011014

\end{itemize}



\cleardoublepage
\pagenumbering{arabic}

\abstract
Manual grading of handwritten exam papers is time-consuming, error-prone, and difficult to scale in educational settings. This project presents an automated optical character recognition (OCR)–based system designed to extract and interpret handwritten content from student exam papers in order to support automated grading and rapid feedback. The proposed approach applies image preprocessing techniques such as paper detection, cropping, noise reduction, and deskewing to improve input quality, followed by segmentation of relevant answer regions. Handwritten text recognition is performed using state-of-the-art OCR models optimized for handwritten data. The extracted text is then normalized to reduce recognition errors and enable further evaluation or comparison with expected answers. The system focuses on short answers and numeric responses commonly found in exam papers, where automation provides the greatest benefit. The results demonstrate that combining targeted preprocessing with modern OCR models can significantly improve recognition reliability for handwritten exam documents. This application aims to reduce instructors’ grading workload, increase consistency in evaluation, and enable faster feedback for students, illustrating the practical potential of OCR technologies in educational assessment workflows.\cleardoublepage

\section{Introduction}
A brief introduction about your project which includes aim and objectives. 

\section{Literature Review}
List and explain briefly related works that have been introduced so far which are similar to your proposed solution/techniques/methodology. Indicate weak and strength sides of the related studies and clarify novelty of your proposed approach.

Cite related studies in a bibtex format. You can find bibtex citations for related studies from the academic websites such as "Google Scholar" and add to "literature.bib" file.   

You can cite any paper in a text like: \cite{corke2011robotics} introduce an approach.....
\cite{mahonyKC12} propose a method for....

\section{Methodology}
Description of the proposed methodology and/or experimental method step by step.

\section{Preliminary Experiments \& Results}
Show and explain preliminary experiments that have done so far with their corresponding setups and results. 

\section{Weekly Schedule/Project Plan}
On the Gantt Chart that you prepared for project proposal report, show the steps that have been already done/undone with a color-coded representation.(green for done steps and red for undone ones.)

\bibliographystyle{chicago}
\bibliography{references}

\end{document}
