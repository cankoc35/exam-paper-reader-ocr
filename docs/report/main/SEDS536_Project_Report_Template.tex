\documentclass[14pt]{article}

\usepackage{graphics}
\usepackage{epsfig}
\usepackage{times}
\usepackage{amsmath}
\usepackage{subcaption}
\usepackage{multirow}
\usepackage{gensymb}
\usepackage{array}
\usepackage{float}
\usepackage{indentfirst}
\newcolumntype{P}[1]{>{\centering\arraybackslash}p{#1}}
\usepackage{booktabs, makecell}
\usepackage{diagbox}
\usepackage{float}
\floatstyle{plaintop}
\restylefloat{table}
\usepackage{natbib}


\topmargin      0.0in
\headheight     0.0in
\headsep        0.0in
\oddsidemargin  0.0in
\evensidemargin 0.0in
\textheight     9.0in
\textwidth      6.5in

\title{{\bf SEDS536 Image Recognition \\ Project Progress Report \\ Spring 2025} \\
	\it Project Title: Handwritten Exam Paper Reader}


\date{\today}

\begin{document}

\pagestyle{plain}
\pagenumbering{roman}
\maketitle


\begin{figure}[H]
	\centering
	\includegraphics[ scale = 0.35]{iyte_logo} 
\end{figure}


\begin{itemize}
	\item Mustafacan Koç - 323011014

\end{itemize}



\cleardoublepage
\pagenumbering{arabic}

\abstract
A summary of the scope and significance of the project, the methodology / techniques used, the results gained / outcomes generated and the conclusions obtained so far. Abstracts are generally a single paragraph and less than 250 words.
\cleardoublepage

\section{Introduction}
A brief introduction about your project which includes aim and objectives. 

\section{Literature Review}
List and explain briefly related works that have been introduced so far which are similar to your proposed solution/techniques/methodology. Indicate weak and strength sides of the related studies and clarify novelty of your proposed approach.

Cite related studies in a bibtex format. You can find bibtex citations for related studies from the academic websites such as "Google Scholar" and add to "literature.bib" file.   

You can cite any paper in a text like: \cite{corke2011robotics} introduce an approach.....
\cite{mahonyKC12} propose a method for....

\section{Methodology}
Description of the proposed methodology and/or experimental method step by step.

\section{Preliminary Experiments \& Results}
Show and explain preliminary experiments that have done so far with their corresponding setups and results. 

\section{Weekly Schedule/Project Plan}
On the Gantt Chart that you prepared for project proposal report, show the steps that have been already done/undone with a color-coded representation.(green for done steps and red for undone ones.)

\bibliographystyle{chicago}
\bibliography{references}

\end{document}


